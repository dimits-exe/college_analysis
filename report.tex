\documentclass[12pt]{article}

\usepackage{enumerate}
\usepackage[utf8]{inputenc}
\usepackage[greek, english]{babel}
\usepackage{alphabeta}
\usepackage{libertine}
\usepackage{graphicx}
\usepackage{biblatex}
\usepackage{wrapfig}
\usepackage{hyperref}
\usepackage[table]{xcolor}

\pagenumbering{arabic}


\newcommand{\linkstyle}[1]{\color{blue}\underline{#1}}

\graphicspath{ {./resources/} }

%\addbibresource{refs.bib}


\title{\Huge A Comprehensive Study on US College Admissions  }

\author{\large Tsirmpas Dimitris\\Athens University of Economics and Business\\Department of Informatics}


\begin{document}
	
	\begin{titlepage}
		\maketitle
		\begin{center}
					
			\includegraphics[width=1\textwidth]{aueb_logo.jpg}
			
			\large Athens University of Economics and Business
			
			\large Department of Statistics
			
			\large Greece
		\end{center}
	
	\end{titlepage}
	
	\tableofcontents
	\newpage
	
	\section{Abstract}
	US College Admissions have and continue to be a subject of great debate among scholars and analysts. Such educational institutions have an interest in selecting the most qualified applicants using limited data, while the applicants themselves often protest admission requirements, especially those deemed discriminatory in nature. This study aims to analyze the factors that contribute to successful admissions in US colleges and universities by comparing their performance on standardized tests.
	
	
	\section{Introduction}
	The study uses a random sample of 200 students who applied to continue their studies in their respective universities. Their application consisted of three standardized tests testing their skills and knowledge in mathematics, social studies and creative writing. The dataset we used is available at LINK.  An overview of the data contained can be found in Table \ref{tab::dataset}.
	
	We make the assumption that the dataset has been acquired through random, unbiased sampling. We also make the assumption that the records using different IDs represent different students.
	
	The study is structured as follows. In Section we make general observations about our data and we form our first hypotheses. In Section \ref{sec::results} we follow up with robust analyses and regression models to prove/disprove these hypotheses. Section \ref{sec::conclusions} follows with an overview and discussion about our findings. Finally, in Section \ref{sec::addendum} we include graphs, tables and supporting documents.
	
	\begin{table}[h]
		\centering
		\rowcolors{2}{gray!25}{white}
		\begin{tabular}
			{ |p{2cm} p{2cm} p{4.5cm} p{4.5cm}| }
			\hline
			\textbf{Name} & \textbf{Type} & \textbf{Description} & \textbf{Range}\\
			\hline
			Id  & Nominal & The student's ID & [1-200] \\
			Gender  & Binary & The student's gender & \{male, female\} \\
			Race  & Nominal & The student's race & \{white, latin-american, asian, african-american\} \\
			Schtype  & Binary & The type of the student's secondary education institution & \{public, private\} \\
			Prog  & Nominal & The student's previous study cycle  & \{general, vocation, academic \} \\
			Write  & Numeric & The grade on the writing test  & [0-100] \\
			Math  & Numeric & The grade on the mathematics test  & [0-100] \\
			Socst  & Numeric & The grade on the social studies test & [0-100] \\
			\hline
		\end{tabular}
		\caption{An overview of the data used in this study.}
		\label{tab::dataset}
	\end{table}

	\section{Results}
	\label{sec::results}
	
	\section{Conclusions \& Discussion}
	\label{sec::conclusions}
	
	\section{Addendum}
	\label{sec::addendum}
	
	
\end{document}